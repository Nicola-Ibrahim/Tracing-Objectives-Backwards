%% Global macros for the inverse decision thesis
%
% This file defines shorthand notations and symbols that are used across
% multiple chapters of the thesis.  Defining frequently used constructs
% here allows you to adjust notation centrally.  The following macros are
% deliberately minimal; feel free to extend them as needed.

% Common sets
\newcommand{\R}{\mathbb{R}}          % Real numbers
\newcommand{\N}{\mathbb{N}}          % Natural numbers
\newcommand{\E}{\mathbb{E}}          % Expectation operator

% Bold vectors and matrices
\newcommand{\vect}[1]{\mathbf{#1}}    % Bold vector
\newcommand{\mat}[1]{\mathbf{#1}}     % Bold matrix

% Calligraphic symbols
\newcommand{\calX}{\mathcal{X}}      % Decision space
\newcommand{\calY}{\mathcal{Y}}      % Objective space

% Operators
\DeclareMathOperator*{\argmin}{arg\,min}
\DeclareMathOperator*{\argmax}{arg\,max}

% Distance shorthand
\newcommand{\dist}{d} % Generic distance function in objective space

% Placeholder figure environment
% Use \placeholderfigure{caption} to insert a framed box in place of an
% illustration.  When you have an actual image, replace the contents of
% this macro with \includegraphics.
\newcommand{\placeholderfigure}[1]{%
  \begin{figure}[t]
    \centering
    \fbox{\parbox[c][6cm][c]{10cm}{\centering #1}}
    \caption{#1}
  \end{figure}
}