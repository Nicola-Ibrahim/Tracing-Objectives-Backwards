%% some booleans 
\newboolean{WorkIsConfidential}


% % % % Language and encoding

% Allow the complete utf8 character set. Be careful so save all files in utf8 encoding. This is especially important for German umlauts.
\usepackage[utf8]{inputenc}

% Allows to add conditional text based on language
\usepackage{iflang}

% KOMA-Script offers scrhack to fix certain well-known problems with a list of packages.
% https://tex.stackexchange.com/q/73248/7396
\usepackage{scrhack}


% % % % Page layout related features

% Headers and footers should now be created with scrlayer-scrpage. It superseeds scrpage2 which superseeds fancyhdr.
% See page 237 of the following documentation:
% http://ftp.fau.de/ctan/macros/latex/contrib/koma-script/doc/scrguien.pdf
\usepackage{scrlayer-scrpage}

% Delete predefined header and footer
\clearscrheadfoot

% When this two commands are not used, the header in the appendix shoes "Inhaltsverzeichnis" instead of "Appendix".
% It is still unclear, why this happens.
\renewcommand{\chaptermark}[1]{\markboth{#1}{}}
\renewcommand{\sectionmark}[1]{\markright{-~#1}{}}

\ihead{\leftmark}
\KOMAoption{headsepline}{0.4pt}
\cfoot[\pagemark]{}
\ohead[]{\pagemark}

% Disables header and footer in italic
\setkomafont{pageheadfoot}{\normalfont}

\usepackage[top=3.0cm, bottom=3.0cm, left=2.5cm, right=2.0cm]{geometry} %optional: includefoot


% % % % Text layout related features

% Subliminal refinements towards typographical perfection
\usepackage{microtype}

\usepackage{xspace}
\usepackage{rotating}

% It provides \textcolor{<color>}{<text>} as well as \color{<color>} to switch the color for some give text or until the end of the group/environment. You can get different shades of gray by using black!x as a color where x is a number from 0 to 100, taken as a percentage.
\usepackage{xcolor}
\newcommand{\inred}[1]{\xspace{\color{red} #1}\xspace} % Use this command to make annotations, which will be removed in the final document.


% % % % Math related features

% Added by Schmalenstroer. Still unclear what both packages do.
\usepackage[italic]{mathastext}
\usepackage{isomath}

% The bm package is loaded, but you should still use \mathbf and \boldsymbol to remain compatible with MathJax.
\usepackage{amssymb, amsmath, bm}

% This package is the current package for units and outperforms siunits and similar packages
\usepackage[output-decimal-marker=comma, exponent-product=times]{siunitx} % German notation
% \usepackage[exponent-product=times]{siunitx} % American notation


% % % % Source code related features

% Put in source code into your document, this especially allows inline source code with \lstinline!example!, where "!" can be exchanged by any delimiter
\usepackage{listings}
\lstset{
    basicstyle=\ttfamily
}

% Pseudo code
\usepackage{algorithm}
\usepackage{algpseudocode}
% Allow one-line if conditions
\newcommand{\LineIf}[3]{ \State \algorithmicif\ {#1}\ \algorithmicthen\ {#2} \algorithmicend\ \algorithmicif }


% % % % Image and plot related features

% Use TikZ to draw images
\usepackage{pgfplots} % Import der Plots aus Matlab
\newlength\fheight % Plots aus Matlab immer gleich gross
\newlength\fwidth % Plots aus Matlab immer gleich gross
\usepackage{tikz}
\pgfplotsset{compat=1.9}

% Precompile TikZ pictures for speedup, this does not seem to work in Poolroom yet
\usetikzlibrary{external}
% \tikzexternalize % Activate, works only with pdflatex

% Subfigure is deprecaded and subfig is the new version of subfigure
% Subcaption works with hyperref whereas subfigure is reported to be broken
\usepackage{caption}
\usepackage{subcaption}

% Use this package to import Inkscape generated pdfs with pdf_tex files
\usepackage{import}

% Macros to adjust position text in Inkscape files. Place the anchor exactly in the middle of the position where the text should appear
% Use \adjTxtPos{multiline\\text}
% or \adjTxtPosRot{lorem\\ipsum}
\newlength{\myheight}
\newcommand{\adjTxtPos}[1]{\settoheight{\myheight}{Fg}{\parbox[c]{\textheight}{\centering \vspace{\myheight}\phantom{F} #1 \\ \phantom{g} }}}
\newcommand{\adjTxtPosRot}[1]{\begin{rotate}{90}\hspace{-0.5\textheight}\adjTxtPos{#1}\end{rotate}}


% % % % Table related features

% Define new column types to allow centering and width of each cell
\usepackage{array}
\newcolumntype{L}[1]{>{\raggedright\let\newline\\\arraybackslash\hspace{0pt}}m{#1}}
\newcolumntype{C}[1]{>{\centering\let\newline\\\arraybackslash\hspace{0pt}}m{#1}}
\newcolumntype{R}[1]{>{\raggedleft\let\newline\\\arraybackslash\hspace{0pt}}m{#1}}

% Tables according to Chicago Manual of Style
\usepackage{tabularx, etoolbox, booktabs}
\usepackage{multirow} % Allows rowspan
\newcommand{\rarray}[1]{\renewcommand{\arraystretch}{#1}} % Slightly enlarged cells
\captionsetup{belowskip=12pt,aboveskip=4pt} % Correct table caption distance


% % % % Links and references setup
%% URL handling
\PassOptionsToPackage{hyphens}{url}
% Allow URLs and other links to be clickable
% Removes borders according to: texdoc hyperref page 4
\usepackage[colorlinks=false, pdfborder={0 0 0}]{hyperref}

% Allow URLs to break at hyphens
\usepackage[hyphens]{url}

% Change style to the style of the normal text instead of ttfamily style.
\urlstyle{same}

% Command for breaking urls into two parts
\newcommand*{\brokenurl}[2]{\href{#1#2}{#1}\par\nopagebreak\href{#1#2}{#2}}



\usepackage{csquotes} % texdoc biblatex: If this package is available, biblatex will use its language sensitive quotation facilities to enclose certain titles in quotation marks.
\usepackage[backend=bibtex, bibencoding=utf8, style=alphabetic, giveninits=true, maxbibnames=99]{biblatex}
% \usepackage[backend=biber, style=ieee]{biblatex} % Alternative
\bibliography{literature}

% According to the glossaries manual, this package should be loaded after hyperref, the toc option makes the glossaries appear in the table of contents
\usepackage[acronym, toc, nonumberlist]{glossaries}
%-------------------------------------------------------------------------------
% Acronyms and abbreviations
%-------------------------------------------------------------------------------

% This file defines acronyms used throughout the thesis.  The glossaries
% package will create an acronym list in the back matter.  Use
% \gls{label} to reference an acronym in the text; the first use will
% expand to the full form with the acronym in parentheses, subsequent
% uses will show only the acronym.

\newacronym{MOP}{MOP}{multi‑objective optimisation problem}
\newacronym{MOO}{MOO}{multi‑objective optimisation}
\newacronym{MOEA}{MOEA}{multi‑objective evolutionary algorithm}
\newacronym{PF}{PF}{Pareto front}
\newacronym{PS}{PS}{Pareto set}
\newacronym{GD}{GD}{generational distance}
\newacronym{IGD}{IGD}{inverted generational distance}
\newacronym{HV}{HV}{hypervolume}
\newacronym{GPR}{GPR}{Gaussian process regression}
\newacronym{RBF}{RBF}{radial basis function network}
\newacronym{RF}{RF}{random forest}
\newacronym{FFNN}{FFNN}{feed‑forward neural network}
\newacronym{SA-MOEA}{SA‑MOEA}{surrogate‑assisted multi‑objective evolutionary algorithm}
\newacronym{IM-MOEA}{IM‑MOEA}{inverse modelling multi‑objective evolutionary algorithm}
\newacronym{NSGA-II}{NSGA‑II}{Non‑dominated Sorting Genetic Algorithm II}
\newacronym{MOEA-D}{MOEA/D}{Multi‑Objective Evolutionary Algorithm based on Decomposition}
% Add this after all new glossary entries
\makeglossaries

% Small font size for captions
\addtokomafont{caption}{\small}

% Serif font for headlines
% https://tex.stackexchange.com/a/25850/7396
\addtokomafont{disposition}{\rmfamily}

% Lines above and below chapter headline
\renewcommand*\chapterheadstartvskip{\noindent\rule{\linewidth}{1pt}\par\vspace{\baselineskip}}
\renewcommand*\chapterheadendvskip{\vspace{\baselineskip}\noindent\rule{\linewidth}{1pt}\par\vspace{\baselineskip}}
