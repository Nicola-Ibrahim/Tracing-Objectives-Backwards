\begin{tikzpicture}[
    % --- Local Definitions for self-contained code ---
    execute at begin picture={
        \definecolor{paretoBlue}{HTML}{0077BB}
        \definecolor{dominatedGray}{HTML}{999999}
        \definecolor{dominanceRegion}{HTML}{EE442F}
        \definecolor{feasibleFill}{HTML}{EDF6F9}
    },
    scale=1.2,
    >=Stealth,
    % Style for the points on the Pareto front
    frontPoint/.style={circle, draw=paretoBlue, fill=paretoBlue, inner sep=2pt, outer sep=0pt},
    % Style for dominated points
    domPoint/.style={circle, draw=dominatedGray!80, fill=dominatedGray!50, inner sep=1.5pt},
    % Style for text labels
    label text/.style={font=\footnotesize\sffamily},
    axis label/.style={font=\small\sffamily\bfseries}
]

    % --- 1. Define Coordinates ---
    % Points on the Pareto Front (forming a convex shape toward origin)
    \coordinate (PF1) at (1, 5.5);
    \coordinate (PF2) at (2, 3.5);
    \coordinate (PF3) at (3.5, 2.2); % We will use this point for the dominance example
    \coordinate (PF4) at (5.5, 1.5);
    \coordinate (PF5) at (7.5, 1.2);

    % Dominated points (scattered "behind" the front)
    \coordinate (D1) at (2.5, 5);
    \coordinate (D2) at (4, 4);
    \coordinate (D3) at (5, 3);
    \coordinate (D4) at (6.5, 2.5);
    \coordinate (D5) at (3, 6);
    \coordinate (D6) at (7, 4);
    \coordinate (D7) at (4.5, 5.5);

    % Axes limits
    \def\xmax{8.5}
    \def\ymax{6.5}


    % --- 2. Draw Feasible / Objective Space (Background Layer) ---
    \begin{pgfonlayer}{background}
        % Shade the region "behind" the front to represent the feasible objective space
        \fill[feasibleFill] plot[smooth tension=0.5] coordinates {(PF1) (PF2) (PF3) (PF4) (PF5)}
            -- (\xmax, 1.2) -- (\xmax, \ymax) -- (1, \ymax) -- cycle;
    \end{pgfonlayer}


    % --- 3. Illustrate Pareto Dominance (Middle Layer) ---
    % We show that PF3 dominates D2 and D3.
    \begin{pgfonlayer}{main}
        % Draw the "Cone of Dominance" originating from PF3
        % Any point in this shaded region is dominated by PF3 (because it's worse in f1 AND f2)
        \fill[dominanceRegion, opacity=0.2] (PF3) rectangle (\xmax, \ymax);
        \draw[dominanceRegion, dashed, thick] (PF3) -- (PF3 |- 0,\ymax) node[above, label text, opacity=1] {Worse $f_1$};
        \draw[dominanceRegion, dashed, thick] (PF3) -- (PF3 -| \xmax,0) node[right, label text, opacity=1] {Worse $f_2$};

        % Add annotations for the dominance example
        \node[label text, anchor=south west, text=dominanceRegion!80!black] at ($(PF3)+(0.2,0.2)$) {Region dominated by Point A};
        \node[frontPoint, label={[label text, text=paretoBlue]below left:Point A}] at (PF3) {};
        \node[domPoint, label={[label text, text=dominatedGray]above right:Point B}] at (D2) {};

        % An arrow showing the relationship explicitly
        \draw[->, thick, dominanceRegion, shorten >=2pt, shorten <=2pt] (PF3) to[bend left=20] node[midway, above right, label text, font=\scriptsize\sffamily] {A dominates B} (D2);
    \end{pgfonlayer}


    % --- 4. Draw Axes ---
    \draw[->, thick, line width=1.2pt] (0,0) -- (\xmax+0.5, 0) node[below left=0.1cm and 0.1cm, axis label, align=center] {Objective 1 ($f_1$)\\$\leftarrow$ Minimize};
    \draw[->, thick, line width=1.2pt] (0,0) -- (0, \ymax+0.5) node[below left=0.1cm and 0.1cm, axis label, rotate=90, align=center] {Objective 2 ($f_2$)\\$\leftarrow$ Minimize};


    % --- 5. Draw the Pareto Front and Points (Foreground Layer) ---
    % Draw the thick line connecting Pareto points
    \draw[paretoBlue, line width=2pt, smooth tension=0.5] plot coordinates {(PF1) (PF2) (PF3) (PF4) (PF5)};

    % Draw dominated points
    \foreach \pt in {D1, D2, D3, D4, D5, D6, D7} \node[domPoint] at (\pt) {};

    % Draw Pareto Front points again (to sit on top of the line)
    \foreach \pt in {PF1, PF2, PF3, PF4, PF5} \node[frontPoint] at (\pt) {};


    % --- 6. Final Annotations ---
    % Label the Pareto Front
    \node[text=paretoBlue, font=\sffamily\bfseries, anchor=south west] at ($(PF5)+(0.2,0.1)$) {Pareto Front (Non-dominated Set)};

    % Label the Feasible Space
    \node[text=dominatedGray!80!black, font=\sffamily\itshape] at (6, 5) {Feasible Objective Space};

\end{tikzpicture}