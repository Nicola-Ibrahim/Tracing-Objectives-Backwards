%-------------------------------------------------------------------------------
% Problem statement 
%-------------------------------------------------------------------------------

\chapter{Problem Statement}\label{chap:problem}

This chapter formalises the inverse decision–mapping problem addressed in this thesis.
We define the task and notation, characterise target admissibility, and specify the
form of the outputs produced by an inverse model.

\section{Problem Setup and Notation}\label{sec:setup}

\textbf{Decision space and constraints.} Let $\mathcal{X}\subset\mathbb{R}^{n}$ denote the
feasible decision space; a decision vector is $\mathbf{x}\in\mathcal{X}$. Feasibility is
governed by inequality constraints $g_j(\mathbf{x})\le 0$ ($j=1,\ldots,J$) and equality
constraints $h_k(\mathbf{x})=0$ ($k=1,\ldots,K$).

\textbf{Outcome space.} Let $\mathcal{Y}\subset\mathbb{R}^{m}$ denote the outcome (objective)
space. The forward map (simulation or measurement) is $\mathbf{f}:\mathcal{X}\to\mathcal{Y}$,
so that $\mathbf{y}=\mathbf{f}(\mathbf{x})$.

\textbf{Data.} We assume access to a dataset
\[
  \mathcal{D}=\{(\mathbf{x}_i,\mathbf{y}_i)\}_{i=1}^{N},\quad \mathbf{x}_i\in\mathcal{X},\ \mathbf{y}_i=\mathbf{f}(\mathbf{x}_i),
\]
obtained from historical runs, simulation sweeps, or exploratory optimisation (not
restricted to Pareto points).

\textbf{Inverse query.} Given a target outcome $\mathbf{y}^{\ast}\in\mathcal{Y}$, the goal is
to infer one or more candidate decisions in $\mathcal{X}$ that (approximately) realise
$\mathbf{y}^{\ast}$ under $\mathbf{f}$.

\paragraph{Ill-posedness.}
Inverse mapping is generally \emph{ill-posed} in the sense of Hadamard: an exact preimage
may not exist (existence), may not be unique (uniqueness), or may be unstable to small
perturbations (stability). Robust formulations therefore rely on regularisation and/or a
probabilistic treatment (details in Chapter~\ref{chap:methodology}).

\section{Reachability and Admissibility}\label{sec:admissibility}

\textbf{Reachable outcomes.} The set of attainable outcomes is
\begin{equation}
  \mathcal{Y}_{\mathrm{reach}}
  \;=\;
  \bigl\{\ \mathbf{f}(\mathbf{x})\ :\ \mathbf{x}\in\mathcal{X},\ g_j(\mathbf{x})\le 0,\ h_k(\mathbf{x})=0\ \bigr\}.
  \label{eq:reachableY}
\end{equation}

\textbf{Inverse solution sets.} For a target $\mathbf{y}^{\ast}$, the exact preimage set is
\begin{equation}
  \mathcal{X}^{\ast}(\mathbf{y}^{\ast})
  \;=\;
  \bigl\{\ \mathbf{x}\in\mathcal{X}\ :\ \mathbf{f}(\mathbf{x})=\mathbf{y}^{\ast},\ g_j(\mathbf{x})\le 0,\ h_k(\mathbf{x})=0\ \bigr\};
  \label{eq:inverseSet}
\end{equation}
the \emph{approximate} preimage set under tolerance $\tau>0$ and distance
$d:\mathcal{Y}\times\mathcal{Y}\to\mathbb{R}_{\ge0}$ is
\begin{equation}
  \mathcal{X}^{\ast}_{\tau}(\mathbf{y}^{\ast})
  \;=\;
  \bigl\{\ \mathbf{x}\in\mathcal{X}\ :\ d\bigl(\mathbf{f}(\mathbf{x}),\mathbf{y}^{\ast}\bigr)\le\tau,\ g_j(\mathbf{x})\le 0,\ h_k(\mathbf{x})=0\ \bigr\}.
  \label{eq:approxInverse}
\end{equation}

\textbf{Admissibility classes.} For a given $\tau$,
\begin{align}
  &\text{Admissible} &&\Longleftrightarrow\quad \mathcal{X}^{\ast}(\mathbf{y}^{\ast})\neq\varnothing
      \quad \text{or}\quad \mathcal{X}^{\ast}_{\tau}(\mathbf{y}^{\ast})\neq\varnothing, \label{eq:adm1}\\
  &\text{Weakly admissible} &&\Longleftrightarrow\quad \mathcal{X}^{\ast}(\mathbf{y}^{\ast})=\varnothing
      \ \text{and}\ \mathcal{X}^{\ast}_{\tau}(\mathbf{y}^{\ast})\neq\varnothing, \label{eq:adm2}\\
  &\text{Inadmissible} &&\Longleftrightarrow\quad \mathcal{X}^{\ast}_{\tau}(\mathbf{y}^{\ast})=\varnothing. \label{eq:adm3}
\end{align}

\section{Output Forms of the Inverse Model}\label{sec:outputs}

We consider two complementary output types:
\begin{itemize}
  \item \textbf{Deterministic (single decision):} return one representative decision $\widehat{\mathbf{x}}\in\mathcal{X}$ per query $\mathbf{y}^{\ast}$.
  \item \textbf{Probabilistic (distribution over decisions):} return a conditional distribution over decisions, enabling uncertainty quantification and multiple plausible candidates.
\end{itemize}
Training objectives, constraint handling, and inference details are given in Chapter~\ref{chap:06_methodology}.

\section{Research Questions}\label{sec:RQs}
We investigate:
\begin{enumerate}
  \item Inverse querying: How can a user request a new target outcome $\mathbf{y}^{\ast}$, not present among observed outcomes, and obtain inferred candidate decisions $\mathbf{x}$ from available data?
  \item Validation of generated decisions: How can the quality of generated decisions—relative to a specified target outcome—be established using feedback or forward evaluations?
  \item Modelling strategies: Which types of modelling strategies are most effective for learning the inverse relationship from outcome space to decision space?
\end{enumerate}
