%-------------------------------------------------------------------------------
% Conclusion
%-------------------------------------------------------------------------------

\chapter{Conclusion}\label{chap:conclusion}

Inverse decision mapping offers a promising paradigm for exploring
multi‑objective design spaces.  By learning a surrogate inverse
relationship from sampled Pareto pairs, designers can specify desired
objective outcomes and receive plausible decision recommendations
instantaneously.  This thesis has presented a comprehensive treatment of
the inverse decision mapping problem, developed a modular framework for
offline training and online exploration, implemented a prototype system
and evaluated it on benchmark problems and a real‑world case study.

\section{Summary of Contributions}

The main contributions of this work are:

\begin{itemize}
  \item A formal definition of the inverse decision mapping problem in
    multi‑objective optimisation, including criteria for feasibility,
    admissibility and quality of inverse solutions.
  \item A survey of related work on model‑based evolutionary algorithms,
    inverse modelling, surrogate‑assisted optimisation and inverse
    multi‑objective optimisation, situating the thesis within the
    literature~\cite{Gholamnezhad2022,Farias2024}.
  \item A modular methodology comprising an offline phase for data
    generation, normalisation, model training and validation, and an
    online phase for interactive exploration using plausibility checks,
    inverse prediction, forward evaluation and candidate ranking.
  \item Implementation of the framework using open‑source tools
    (\texttt{pymoo}, \texttt{scikit‑learn}, \texttt{Dash}) and its
    application to standard benchmark problems (ZDT, DTLZ) and a
    cantilever beam design case study.
  \item Experimental results demonstrating that Gaussian process
    regression provides the most accurate inverse model, that random
    forests offer a competitive and diverse alternative and that data
    quantity and transfer learning significantly influence performance.
  \item A discussion of benefits, limitations and future directions,
    including the potential for hybrid training, active sampling,
    generative models, human–in‑the‑loop optimisation and distributionally
    robust methods~\cite{Dong2021}.
\end{itemize}

\section{Key Findings}

Our experiments indicate that high‑quality inverse predictions can be
obtained from relatively modest datasets (200–500 samples) and that
uncertainty quantification via Gaussian processes aids in distinguishing
feasible from infeasible targets.  The separation of offline and online
phases makes the framework practical for interactive decision support
systems.  Inverse mapping can reduce the number of expensive forward
evaluations required to identify satisfactory designs, accelerating
design iterations and enabling richer exploration of the trade‑off space.

\section{Future Work}

Future research should further investigate data‑efficient inverse
modelling.  Combining active learning with inverse mapping could focus
sampling on underrepresented regions of the Pareto front.  Deep
generative models and structured multioutput regression warrant
exploration to capture complex inverse relationships.  Integrating
user preferences through preference learning and augmenting plausibility
checks with robust statistical methods will enhance user trust and
personalisation.  Finally, expanding the framework to handle dynamic
objectives, constraints and real‑time feedback from physical systems will
open new application domains.

In summary, this thesis lays the foundation for data‑driven inverse
multi‑objective design exploration.  By uniting insights from multi‑
objective optimisation, surrogate modelling, inverse learning and human–
computer interaction, the presented framework paves the way for more
interactive, efficient and personalised decision support tools.