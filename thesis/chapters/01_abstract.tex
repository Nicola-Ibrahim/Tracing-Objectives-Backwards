%-------------------------------------------------------------------------------
% Abstract
%-------------------------------------------------------------------------------

\chapter*{Abstract}
\addcontentsline{toc}{chapter}{Abstract}

Design problems in engineering and science often involve multiple, conflicting objectives, and while conventional workflows map decisions to outcomes, they offer limited support for the inverse task of identifying decisions that realise a user-specified pattern of objective values. This thesis develops a general, learning-based framework for inverse exploration in multi-objective settings: from representative forward evaluations, we learn a mapping from outcome space back to decision space and use it at query time to propose candidates aligned with target outcomes, keeping the description model-agnostic and allowing interactive refinement without repeated end-to-end optimisation. Evaluation on synthetic benchmarks and a real-world case study shows that the learned inverse map recovers decisions close to the desired targets while substantially reducing computational effort compared with re-running forward optimisation for each new query; overall, the work advances inverse design by formalising inverse decision mapping for multi-objective problems, introducing a practical data-driven workflow for interactive use, and providing empirical evidence that clarifies trade-offs and informs method selection in practice.