%-------------------------------------------------------------------------------
% Abstract
%-------------------------------------------------------------------------------

\chapter*{Abstract}
\addcontentsline{toc}{chapter}{Abstract}

Modern design tasks increasingly demand configurations tailored to a desired performance profile, yet multi-objective settings make this request difficult to answer directly. While forward evaluation maps decisions to objective values, the relationship is often strongly nonlinear, and many-to-one, so the inverse question—which decisions can realise a user-specified pattern of objectives—cannot be resolved by inspection and is expensive to address through unguided trial-and-error in the decision space. This thesis investigates inverse exploration for multi-objective problems by tracing objectives backwards from outcome space to decision space. It develops a machine-learning-based, data-driven approach for inverse decision mapping that learns from previously computed forward evaluations and, at query time, proposes multiple candidate decisions aligned with a target objective vector, enabling iterative refinement without repeated expensive search loops. Experiments on synthetic benchmarks and a real-world case study show that the learned inverse map can return candidates whose forward-evaluated outcomes closely match target specifications, while reducing computational effort when many inverse queries are issued. Overall, the thesis contributes a clear formulation of inverse decision mapping for multi-objective settings, a practical workflow for interactive use, and empirical evidence that characterises trade-offs between target alignment, candidate diversity, and computational cost to guide method selection in practice.

