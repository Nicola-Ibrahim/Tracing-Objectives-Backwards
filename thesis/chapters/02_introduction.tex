%-------------------------------------------------------------------------------
% Introduction
%-------------------------------------------------------------------------------

\chapter{Introduction}

Many design problems in signal processing and related systems are posed in reverse. Practitioners often begin with a desired performance pattern and then ask which controllable settings will produce it. This viewpoint appears across familiar tasks: one may prescribe a frequency-response template and seek filter parameters, specify beam-shape constraints and seek array weights, target an operating point defined by fidelity and resource budgets and seek encoder settings, or work backward from desired detection characteristics to system parameters. In all of these cases, forward evaluation is routine, while inverse design is the bottleneck.

To make this setting precise, let $\calX$ denote the space of controllable decisions $\vect{x}$ and let $\calY$ denote the space of measurable or computable performance descriptors $\vect{y}$. A forward model provides a mapping
\begin{equation}
  f:\calX \to \calY,\qquad \vect{y}=f(\vect{x}),
\end{equation}
implemented by analysis, simulation, or experiment. Inverse design asks the complementary question: given a target pattern $\vect{y}^{\star}\in\calY$, identify decisions $\vect{x}\in\calX$ whose forward outcomes match $\vect{y}^{\star}$ closely enough for the application.


This inverse relationship is rarely a straightforward inversion. Hadamard characterises a problem as well posed when a solution exists, is unique, and depends continuously on the data; inverse problems frequently violate one or more of these conditions~\cite{KabanikhinSurvey}. Instability is especially common: small errors in measured or simulated outcomes, or small specification changes in $\vect{y}^{\star}$, can correspond to large changes in inferred decisions. Practical inverse design therefore requires additional structure to restore robustness, typically through regularisation that encodes preferences and stabilises inference, or through statistical formulations that represent uncertainty and non-uniqueness directly~\cite{BenningBurger2018,Stuart2010}.


Forward-only workflows address inverse questions indirectly by iteratively perturbing $\vect{x}$, re-evaluating the forward model, and repeating until the observed outcomes satisfy the specification. While effective in low-dimensional and cheap-to-evaluate settings, this strategy becomes expensive when forward evaluations are costly and provides limited insight into the inverse relationship itself. In particular, it does not naturally reveal when a target is unattainable, when multiple distinct designs satisfy essentially the same target, or how sensitive a design is to small changes in the specification.

This thesis studies an alternative approach that uses data to make inverse exploration more direct. Given representative examples $\{(\vect{x}_i,\vect{y}_i)\}_{i=1}^N$ generated by the forward model, we learn a decision-mapping surrogate
\begin{equation}
  \hat{g}:\calY \to \calX,
\end{equation}
that can be queried with a target $\vect{y}^{\star}$ to produce candidate decisions intended to realise that target. Modern machine learning models, including neural networks, are well suited to this role because they can approximate complex, high-dimensional nonlinear relationships from data and provide fast proposals at query time. When the inverse relationship is inherently multi-valued, probabilistic and generative modelling ideas provide a natural way to represent families of plausible decisions and to express ambiguity arising from noise, modelling mismatch, or limited data, rather than collapsing the problem to a single design point~\cite{Stuart2010,Ongie2020}.



\begin{figure}[h]
  \centering
  \begin{tikzpicture}
  % ===== Colors (define before use) =====
  \definecolor{panelBlue}{RGB}{33,111,219}   % used for left panel accent
  \definecolor{softGray}{RGB}{246,248,250}   % light chip/bg
  \definecolor{inversePurple}{RGB}{124,77,255} % right panel accent
  \definecolor{textDim}{RGB}{80,80,90}

  % ===== Global styles =====
  \tikzset{
    >={Latex[length=2.6mm]},
    font=\sffamily,
    panel/.style={
      draw, rounded corners=8pt, line width=0.8pt,
      fill=softGray, inner sep=10pt,
      minimum width=7.6cm, minimum height=4.2cm
    },
    title/.style={font=\small\itshape, text opacity=0.9, text=textDim},
    box/.style={
      draw, rounded corners=6pt, line width=0.8pt, fill=white,
      align=center, inner sep=7pt, minimum height=10mm
    },
    circ/.style={
      circle, draw, line width=0.9pt, fill=white, minimum size=9mm, align=center
    },
    cand/.style={box, minimum width=15mm, font=\footnotesize},
    small/.style={box, minimum width=22mm, font=\scriptsize},
    flow/.style={-Latex, line width=0.9pt},
    softflow/.style={-Latex, line width=0.9pt, dashed, opacity=0.7},
    labelchip/.style={
      fill=softGray, rounded corners=3pt, inner sep=1.6pt,
      font=\scriptsize, align=center
    }
  }

  % ===== Anchors for two panels =====
  \coordinate (C) at (0,0);
  \coordinate (L) at ($(C)+(-4.4,0)$);
  \coordinate (R) at ($(C)+( 4.4,0)$);

  % ===================== Left: Forward (evaluation) x -> y =====================
  \node[panel, anchor=center] (PL) at (L) {};
  % subtle accent strip
  \draw[panelBlue!20, line width=6pt, rounded corners=6pt]
       ($(PL.north west)+(6pt,-6pt)$) -- ++(2.1,0);

  \node[title, anchor=west] at ($(PL.north west)+(6pt,-6pt)$)
       {Forward (evaluation): $\mathbf{x}\!\rightarrow\!\mathbf{y}$};

  \node[circ] (xL) at ($(PL.west)+(1.6cm,0)$) {$\mathbf{x}$};
  \node[box, minimum width=27mm] (fbox) [right=12mm of xL] {$\mathbf{y}=f(\mathbf{x})$};
  \node[circ] (yL) [right=12mm of fbox] {$\mathbf{y}$};

  \draw[flow, panelBlue!90!black] (xL) -- (fbox);
  \draw[flow, panelBlue!90!black] (fbox) -- (yL);

  \node[labelchip, anchor=north] at ($(fbox.south)+(0,-3pt)$) {known/simulatable forward};

  % ===================== Right: Inverse (design) y* -> {x} =====================
  \node[panel, anchor=center] (PR) at (R) {};
  % subtle accent strip
  \draw[inversePurple!25, line width=6pt, rounded corners=6pt]
       ($(PR.north west)+(6pt,-6pt)$) -- ++(2.1,0);

  \node[title, anchor=west] at ($(PR.north west)+(6pt,-6pt)$)
       {Inverse (design): $\mathbf{y}\!\rightarrow\!\mathbf{x}$};

  \node[circ] (yR) at ($(PR.west)+(1.45cm,0.0)$) {$\mathbf{y}^{\star}$};
  \node[box, minimum width=30mm] (gbox) [right=12mm of yR] {$\mathbf{x}=\hat{g}(\mathbf{y})$};

  % candidates
  \node[cand] (x1) [right=10mm of gbox, yshift=8mm]  {$\mathbf{x}^{(1)}$};
  \node[cand] (x2) [right=10mm of gbox]              {$\mathbf{x}^{(2)}$};
  \node[cand] (x3) [right=10mm of gbox, yshift=-8mm] {$\mathbf{x}^{(3)}$};

  \draw[flow, inversePurple!80!black] (yR) -- (gbox);
  \draw[flow, inversePurple!80!black] (gbox.east) .. controls ($(gbox.east)+(6mm, 7mm)$) .. (x1.west);
  \draw[flow, inversePurple!80!black] (gbox.east) -- (x2.west);
  \draw[flow, inversePurple!80!black] (gbox.east) .. controls ($(gbox.east)+(6mm,-7mm)$) .. (x3.west);

  \node[labelchip, anchor=north] at ($(gbox.south)+(0,-3pt)$) {learned inverse model};

  % optional forward-check (dashed; helps intuition without clutter)
  \node[small] (fhat) [below=10mm of x2] {$\hat{\mathbf{y}}=\hat{f}(\mathbf{x})$};
  \node[circ, minimum size=8mm, font=\scriptsize] (yChk) [left=10mm of fhat] {$\hat{\mathbf{y}}$};

  \foreach \z in {x1,x2,x3}{
    \draw[softflow] (\z.south) -- ++(0,-5mm) -| (fhat.north);
  }
  \draw[softflow] (fhat.west) -- (yChk.east);
  \draw[softflow] (yChk) |- ($(yR.north)+(0,2mm)$);

  % light guide lines indicating comparison
  \draw[densely dotted, opacity=0.5] (yR.south) -- ++(0,-1.3);
  \draw[densely dotted, opacity=0.5] (yChk.north) -- ++(0,1.3);
\end{tikzpicture}

  \caption{Forward vs.\ inverse viewpoint. Forward mapping evaluates how decisions $\vect{x}$ produce outcomes $\vect{y}$; the inverse task seeks decisions that realise a desired outcome pattern.}
  \label{fig:forward-inverse}
\end{figure}


Inverse questions arise throughout signal-processing practice. A designer may specify a frequency-response template and seek filter parameters, prescribe a spatial beam pattern and seek array weights, target a rate--distortion operating point and seek encoder settings, or work backward from desired detection performance to system parameters. These examples share the same core difficulty: forward evaluation is typically straightforward, while the inverse mapping from a target outcome to feasible decisions can be ambiguous, sensitive, or even infeasible. This makes data-driven models, particularly modern machine learning methods, a natural tool for proposing candidate designs quickly and supporting interactive exploration under uncertainty.


\section{Problem statement and purpose}

We consider inverse decision mapping for multi-criteria signal-processing systems. Assume access to representative examples $\mathcal{D}=\{(\vect{x}_i,\vect{y}_i)\}_{i=1}^N$ with $\vect{x}_i\in\calX$, $\vect{y}_i\in\calY$, and $\vect{y}_i=\vect{f}(\vect{x}_i)$ for a forward map $\vect{f}:\calX\to\calY$. The problem is to learn a data-driven rule that, given a user-specified target $\vect{y}^{\star}\in\calY$, returns one or more candidate decisions $\widehat{\vect{x}}\in\calX$ whose forward evaluations $\vect{f}(\widehat{\vect{x}})$ are close to $\vect{y}^{\star}$ under task-relevant discrepancy measures in objective space. This inverse relationship is generally ill-posed: multiple distinct decisions may correspond to similar outcomes, some targets may have no feasible preimage, and small perturbations in $\vect{y}^{\star}$ or in the observed outcomes can yield large changes in the inferred decision~\cite{KabanikhinSurvey,BenningBurger2018,Stuart2010}.

The purpose of this thesis is to develop and assess a principled approach for inverse decision mapping that is transparent, reproducible, and suitable for interactive use across application domains. The approach uses machine learning models as fast proposal mechanisms for candidate decisions, while treating forward evaluation as the reference for verification and for application-aligned assessment. This combination reflects a broader direction in inverse problems research that leverages data-driven models to accelerate inverse querying without abandoning forward-model grounding.



\section{Research questions}

\begin{enumerate}
  \item Inverse querying. How can a new target $\vect{y}^{\star}$ be converted into one or more concrete candidate decisions $\widehat{\vect{x}}$ using available examples and forward evaluations, while accounting for non-uniqueness and potential infeasibility?
  \item Quality of generated decisions. How should the quality of $\widehat{\vect{x}}$ be assessed against $\vect{y}^{\star}$ using forward verification in a way that is explicit, application-aligned, and reproducible across experiments?
  \item Modelling strategies and generalisation. Which modelling choices most effectively learn mappings from outcomes to decisions, and under what data conditions and problem characteristics do they generalise reliably to unseen targets?
\end{enumerate}



\section{Assumptions, limitations, and delimitations}

We assume access to a finite, publicly available collection of representative pairs
$\mathcal{D}=\{(\vect{x}_i,\vect{y}_i)\}_{i=1}^N$ generated by a forward process
$\vect{y}=\vect{f}(\vect{x})$ (or by a trusted proxy for $\vect{f}$), and we assume that this dataset provides meaningful coverage of the outcome region in which inverse queries are expected. We further assume that outcomes can be compared with a task-relevant discrepancy measure $d_{\calY}(\cdot,\cdot)$, and that candidate decisions can be restricted to the admissible domain through simple parameter bounds or projection when needed.

The primary limitation is coverage: inverse queries that lie outside, or near the boundary of, the outcome region represented in $\mathcal{D}$ may induce extrapolation and unreliable proposals. A related limitation is distribution shift between the published data used for training and the targets posed at deployment, which can degrade inverse accuracy even when the forward relationship itself is unchanged. Finally, the inverse relationship is inherently non-unique in many problems, so any learned inverse model may return only a subset of the feasible decision set and may miss rare or disconnected solution families; generating multiple candidates mitigates this but does not remove the underlying ambiguity.

We delimit the empirical scope to a small set of benchmarks with two-dimensional outcome descriptions and to publicly released datasets of manageable size, chosen to support controlled comparison and reproducible evaluation. The proposed modelling and evaluation principles are stated to extend beyond these delimitations, but claims in this thesis are supported only within the studied settings.

\section{Methodological overview}

The study uses publicly available decision--outcome examples to train a model that supports inverse querying. From a dataset $\mathcal{D}=\{(\vect{x}_i,\vect{y}_i)\}_{i=1}^N$ with $\vect{y}_i=\vect{f}(\vect{x}_i)$, we apply consistent preprocessing to $\calX$ and $\calY$ (including scaling where appropriate) and learn a mapping from target outcomes to candidate decisions. Since the inverse relationship is commonly one-to-many, the model is used to generate a set of candidates for a single target: for a query $\vect{y}^{\star}$ it returns $\{\widehat{\vect{x}}^{(k)}\}_{k=1}^K$, where $K$ controls the number of proposals and diversity arises from model stochasticity or sampling.

Candidate quality is established through forward verification. For each proposal, we compute or approximate $\vect{f}(\widehat{\vect{x}}^{(k)})$ and measure discrepancy to the target using a task-relevant outcome-space distance $d_{\calY}(\vect{f}(\widehat{\vect{x}}^{(k)}),\vect{y}^{\star})$. The method returns one or more candidates according to an explicit selection rule (for example, the lowest-discrepancy proposal), and evaluation on held-out targets reports both discrepancy and tolerance-based success. Because the output is a set, primary performance summaries are reported in set form, such as the best-achieved discrepancy over $K$ candidates per target, together with statistics that show how performance varies with $K$ and with target location relative to the outcome region covered by the data.


% ------------------------------------------------------------------
% \section{Contributions}

% \begin{itemize}
%   \item \textbf{Problem formulation.} We formalise inverse decision mapping in multi-objective
%   settings, treating the inverse as an ill-posed inference task in the sense of Hadamard
%   (non-uniqueness, instability, possible non-existence). This framing clarifies when targets
%   may be unattainable or ambiguous and motivates regularisation/statistical treatments for
%   meaningful reconstruction of decisions from desired outcomes. \cite{KabanikhinSurvey}

%   \item \textbf{Comprehensive background.} We survey core concepts in multi-objective
%   optimisation (problem structures, algorithmic families, and performance assessment)
%   and review inverse-problem perspectives that inform stable learning of inverse maps.
%   The survey integrates established MOO tutorials and modern views on regularisation and
%   Bayesian inference for inverse problems. \cite{BenningBurger2018,Stuart2010}

%   \item \textbf{General modelling approach.} We propose a data-driven approach to inverse
%   exploration that learns a mapping $\hat{g}\colon\mathcal{Y}\!\to\!\mathcal{X}$ from representative
%   examples (e.g., prior designs, historical runs, or exploratory evaluations). At query time,
%   the method returns candidate decisions aligned with a user-specified objective pattern,
%   with optional forward evaluation for verification/refinement. The approach is agnostic to
%   how examples are obtained and does not rely on a specific optimisation paradigm.

%   \item \textbf{Deterministic and probabilistic models.} We develop and compare deterministic
%   and probabilistic strategies for inverse exploration. Deterministic models emphasise
%   parsimony and speed, using regularisation to counteract ill-posedness; probabilistic models
%   represent distributions over decisions to expose uncertainty and multi-valued relationships.
%   We highlight when each perspective is advantageous and how they can be combined in practice.

%   \item \textbf{Empirical evaluation.} Experiments on synthetic benchmarks and a real-world
%   case study show that learned inverse mappings can recover decision candidates that match
%   desired objective patterns while substantially reducing the number of forward evaluations
%   compared with repeated forward search. We analyse accuracy–effort trade-offs and compare
%   models across training-set sizes.
% \end{itemize}

% ------------------------------------------------------------------
\section{Structure of the Thesis}

The remainder of this thesis is organised as follows.

Chapter~\ref{chap:motivation} provides background on multi-objective
optimisation and inverse problems: it introduces core MOO concepts and
performance assessment at a high level, reviews Hadamard’s well-posedness
criteria and why inverse mappings are often ill-posed, and motivates a
data-driven inverse viewpoint for design.

Chapter~\ref{chap:related} surveys related work on learning inverse mappings
from examples, including modelling strategies,
surrogate-assisted and model-based approaches, and interactive decision support
systems. Emphasis is placed on methods that are agnostic to data sources and
optimisation paradigms.

Chapter~\ref{chap:problem} formalises the inverse decision-mapping problem,
clarifies assumptions (identifiability, stability), and states the research
questions and evaluation criteria used throughout the thesis.

Chapter~\ref{chap:method} presents the proposed approach for inverse
exploration. We describe how representative \((\vect{x},\vect{y})\) examples
can be collected from historical runs, simulations, or exploratory evaluations;
how a mapping \(\hat{g}\colon \mathcal{Y}\!\to\!\mathcal{X}\) is learned; and how
query-time proposals are generated and optionally verified by forward
evaluation—without committing to a specific optimisation family.

Chapter~\ref{chap:implementation} outlines practical aspects of the reference
implementation used for experiments (data handling, model training/selection,
querying, and visual diagnostics), with an emphasis on modular components.

Chapter~\ref{chap:experiments} reports results on synthetic benchmarks and a
real-world case study, comparing deterministic and probabilistic inverse
models, analysing accuracy–effort trade-offs, and quantifying reductions in
forward evaluations relative to repeated forward search.

Chapter~\ref{chap:discussion} synthesises the empirical findings, discusses
limitations and threats to validity, and reflects on when each modelling
perspective is advantageous. Finally, Chapter~\ref{chap:conclusion} concludes
and outlines avenues for future work.
