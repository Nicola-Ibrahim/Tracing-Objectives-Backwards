%-------------------------------------------------------------------------------
% Background and Motivation
%-------------------------------------------------------------------------------

\chapter{Background and Motivation}\label{chap:motivation}

Inverse decision mapping builds on ideas from multi-objective optimisation (MOO)
and inverse problems. This chapter introduces (i) a compact view of MOO and
trade-off sets, (ii) the forward–inverse perspective and why inverse problems
are often ill-posed, and (iii) practical ways to obtain representative
\((\vect{x},\vect{y})\) pairs for learning an inverse map. We keep the
discussion agnostic to any specific optimiser; Pareto-based tools are one
useful option among several. For broad MOO fundamentals and assessment, see
Deb~\cite{Deb2011Intro} and the tutorial by Emmerich \& Deutz~\cite{Emmerich2018Tutorial}.

\section{Multi-Objective Optimisation}

A multi-objective optimisation problem (MOP) is defined by a vector of decision variables \(\vect{x}=(x_1,\dots,x_n)\in\calX\subset \R^n\) and a vector of objective functions \(\vect{f}=(f_1,\dots,f_m)\colon \calX\to \calY\subset \R^m\). Without loss of generality we consider all objectives to be minimised. A MOP can be written compactly as

\begin{align} 
\min_{\vect{x}\in\calX}\; \vect{f}(\vect{x}) &= \big(f_1(\vect{x}),\dots,f_m(\vect{x})\big),\\ \text{s.t.}\; g_j(\vect{x}) &\le 0,\quad j=1,\dots,J,\\ h_k(\vect{x}) &= 0,\quad k=1,\dots,K, 
\label{eq:mop}
\end{align}


where \(g_j\) and \(h_k\) denote inequality and equality constraints, respectively. The feasible region \(\calX\) is assumed to be a compact, connected subset of \(\R^n\). Each objective \(f_i\) may be non-convex, nonlinear and expensive to evaluate. Because objectives typically conflict, the goal is a \emph{set of trade-off
designs} rather than a single optimum. A standard partial order uses
\emph{Pareto dominance}: for \(\vect{u},\vect{v}\in\R^m\),
\(\vect{u}\preceq\vect{v}\) iff \(u_i\le v_i\ \forall i\) and \(u_j<v_j\) for
some \(j\). A feasible \(\vect{x}^\star\) is Pareto-optimal if no other
\(\vect{x}\) satisfies \(\vect{f}(\vect{x})\preceq \vect{f}(\vect{x}^\star)\).
See the references above for a concise survey of algorithmic families and
assessment practices.
MOPs arise in a broad spectrum of application domains, including engineering, economics, logistics, manufacturing and science~\cite{Tan2023}. In practice, objective functions often conflict: improving one performance metric deteriorates another. 



\section{Forward and Inverse Viewpoints}

The \emph{forward} process evaluates how decisions produce outcomes:
\begin{equation}
  \vect{y} \;=\; \vect{f}(\vect{x}) \;+\; \boldsymbol{\varepsilon},
  \qquad \vect{x}\in\calX,\ \vect{y}\in\calY,
  \label{eq:forward}
\end{equation}
where \(\boldsymbol{\varepsilon}\) represents measurement/model error. Inverse
exploration asks for decisions that realise a desired outcome pattern
\(\vect{y}^{\star}\): find \(\vect{x}\) with \(\vect{f}(\vect{x})\approx
\vect{y}^{\star}\). Inverse problems often fail Hadamard’s well-posedness
criteria (existence, uniqueness, stability) because many distinct decisions
can map to similar outcomes and small output errors can induce large changes
in the inferred input—i.e., they are \emph{ill-posed}. \cite{KabanikhinSurvey,Stuart2010}

Two complementary formalisms are widely used to make the inverse task
meaningful and robust:

\paragraph{Regularised optimisation.}
Seek \(\vect{x}\) by solving
\begin{equation}
  \widehat{\vect{x}}
  \;\in\; \arg\min_{\vect{x}\in\calX}
  \underbrace{\mathcal{L}\big(\vect{f}(\vect{x}),\,\vect{y}^{\star}\big)}_{\text{data-fit}}
  \;+\;
  \lambda\,\underbrace{R(\vect{x})}_{\text{regulariser}},
  \label{eq:tikhonov}
\end{equation}
where \(\mathcal{L}\) measures mismatch (e.g., squared or Mahalanobis distance)
and \(R\) encodes prior structure (e.g., smoothness, sparsity, bounded
complexity). Regularisation restores stability to ill-posed inverses by
penalising implausible solutions. \cite{BenningBurger2018}

\paragraph{Probabilistic/Bayesian formulation.}
Treat \(\vect{x}\) as random and infer the posterior
\begin{equation}
  p(\vect{x}\mid \vect{y}^{\star})
  \;\propto\;
  \underbrace{p(\vect{y}^{\star}\mid \vect{x})}_{\text{likelihood}}
  \;\underbrace{p(\vect{x})}_{\text{prior}},
  \label{eq:posterior}
\end{equation}
which quantifies uncertainty and captures multi-valued relationships between
decisions and outcomes. Point estimates (MAP, posterior mean) and samples
provide candidate designs and uncertainty summaries. \cite{Stuart2010}

In this thesis we adopt both \emph{deterministic} (single-solution, typically
regularised) and \emph{probabilistic} (distributional) models as complementary
tools for inverse exploration, in line with standard inverse-problems practice.
\cite{BenningBurger2018,Stuart2010}

\section{DataBase}

Learning a data-driven inverse map \(\hat{g}:\calY\!\to\!\calX\) requires a set
\(\mathcal{D}=\{(\vect{x}_i,\vect{y}_i)\}_{i=1}^{N}\) with \(\vect{y}_i=
\vect{f}(\vect{x}_i)\). In practice, \(\mathcal{D}\) can be built from:
\begin{itemize}
  \item Historical runs and prior designs (e.g., archives, logs).
  \item Simulation sweeps/DoE that span \(\calX\) and reveal trade-offs.
  \item Exploratory optimisation, including evolutionary, decomposition-
        based, or surrogate-assisted methods that generate diverse samples.
\end{itemize}
Our approach is \emph{agnostic} to how \(\mathcal{D}\) is obtained; it consumes
input–output examples without relying on a single optimisation family. For
surveys covering these algorithmic paradigms and assessment, see
Deb; Emmerich \& Deutz. \cite{Deb2011Intro,Emmerich2018Tutorial}


\section{Towards Inverse Design}\label{sec:inverse_data_eval}

Most materials workflows are built for the \emph{forward} question $\,\vect{x}\!\mapsto\!\vect{y}$—given a candidate, what will it do? Design asks the converse: specify a target response $\vect{y}^{\star}$ and identify admissible decisions $\vect{x}$ that realise it under physical and process constraints. Because many $\vect{x}$ can yield similar $\vect{y}$, the inverse task is generically ill-posed. We therefore pose it explicitly as a regularised/Bayesian inverse problem (Eqs.~\ref{eq:tikhonov}–\ref{eq:posterior}), which affords (i) fast, feasible point proposals consistent with prior knowledge and constraints, and (ii) calibrated distributions over plausible designs that make non-uniqueness and uncertainty first-class citizens~\cite{BenningBurger2018,Stuart2010}.

An inverse model proposes decisions $\vect{x}\in\mathcal{X}$ that are likely to realise a desired outcome pattern $\vect{y}^{\star}\in\mathcal{Y}$. Rather than tying learning to a specific data-generation protocol (e.g., forward Pareto sweeps), we treat \emph{inverse exploration} as a data-driven task that leverages any representative set $\mathcal{D}=\{(\vect{x}_i,\vect{y}_i)\}_{i=1}^{N}$—from historical logs, simulation sweeps, or exploratory optimisation—together with principled modelling (deterministic or probabilistic) to handle ill-posedness (non-uniqueness, instability) and quantify uncertainty where appropriate. This view aligns with modern treatments of multi-objective optimisation and inverse problems~\cite{Emmerich2018Tutorial,Miettinen1999,Stuart2010,BenningBurger2018}.

For clarity, we group assumptions into: (i) \emph{structural} assumptions on the inverse map $g:\mathcal{Y}\!\to\!\mathcal{X}$ (e.g., smoothness, sparsity, Lipschitz bounds), (ii) \emph{feasibility} constraints $g_j(\vect{x})\!\le\!0$, $h_k(\vect{x})\!=\!0$ inherited from the design domain (cf.\ Eq.~\eqref{eq:mop}), and (iii) \emph{data} assumptions (coverage, noise models). When many $\vect{x}$ produce similar $\vect{y}$, the inverse is non-unique; well-posedness is recovered either by regularisation or by adopting a probabilistic posterior/predictive view that represents distributions over $\vect{x}$ conditional on $\vect{y}$.\footnote{See \cite{Stuart2010,KaipioSomersalo2005,Tarantola2005} for detailed treatments of existence/uniqueness/stability and statistical formulations.} In practice, training and query distributions may differ (dataset/covariate shift), motivating validation protocols and uncertainty reporting that are explicit about the deployment setting~\cite{QuinoneroCandela2009}.

Taken together, this framing turns the inverse task into \emph{guided exploration} of a constrained, often multi-modal design space, where priors, regularisation, and uncertainty are first-class citizens. The next section organises this exploration into a small set of complementary strategy families.


\subsection{Inverse-design strategies}

Inverse design starts from a target response and searches the (typically vast and constrained) space of compositions, structures, and processes for feasible solutions. Compared with forward modeling, inverse problems are often ill-posed (one-to-many mappings) and data-limited, amplifying concerns around generalization, uncertainty, and reproducibility in machine-learning (ML) workflows~\cite{Butler2018_Nature,Wei2019_InfoMat,Draxl2019_JPhysMater}. In practice, three complementary strategy families are prevalent: \textbf{exploration-based} (agent- or heuristic-driven search with minimal labels), \textbf{model-based} (learning bidirectional or generative structure–property surrogates), and \textbf{optimization-based} (treating inverse design as an explicit optimization over design variables). Below we outline their core ideas, typical algorithms, representative materials applications, and practical trade-offs.

\subsubsection{Exploration-based inverse design:}
Exploration-first approaches iteratively probe unknown design regions to discover candidates that satisfy performance constraints under sparse supervision. Reinforcement learning (RL) agents optimize sequences of design decisions by interacting with a simulator or learned environment; Monte Carlo Tree Search (MCTS) supplies strong look-ahead priors; and particle-swarm methods provide robust heuristic global search. For photonic device geometry, Hwang \emph{et~al.} combined an Advantage Actor–Critic (A2C) agent (IDEA) with a critic-value branch tree (CVBT) to diversify high-scoring candidate designs~\cite{Hwang2022_ASOC}. CASTING extends MCTS to continuous actions for materials discovery, improving exploration efficiency in high-dimensional spaces~\cite{Banik2023_NPJCM}, while MCTS has also accelerated polymer sequence design under astronomical combinatorics~\cite{Patra2020_Nanoscale}. When physics solvers are expensive or gradients are unavailable, population heuristics such as Particle Swarm Optimization (PSO) remain effective global explorers; notably, PSO coupled with self-consistent field theory (SCFT) has realized bulk copolymer morphologies meeting target patterns~\cite{Khadilkar2017_Macromolecules,Wang2018_SoftComput}. RL has likewise guided combinatorial chemistry to reach molecules with extreme property targets that confound distribution-learning models~\cite{Kim2024_ChemSci}. Strengths of exploration methods include label frugality, no need for bijective mappings, and natural diversity; typical limitations are high compute budgets (many simulator calls), reward shaping sensitivity, and difficulty enforcing hard constraints without additional mechanism design.

\subsubsection{Model-based inverse design.}
Model-based strategies learn data-driven or physics-informed maps that enable inference from desired properties back to candidate designs. A pragmatic pattern is \emph{forward–inverse coupling}: Liu \emph{et~al.} introduced a tandem neural architecture that mitigates non-uniqueness by training an inverse network cascaded with a forward predictor, so only property-consistent designs survive~\cite{Liu2018_ACSPhotonics}. Industrial alloy workflows have operationalized this idea: the ML Design System (MLDS) generates property$\to$composition (P2C) proposals that are validated by a more reliable composition$\to$property (C2P) forward model; iterations continue until target errors are met, enabling copper and aluminum alloy discoveries~\cite{Wang2019_NPJCM,Jiang2022_JMST}. 
Surrogate modeling is the other mainstay: random-forest or neural surrogates approximate expensive simulators and are inverted with classical optimizers. In Li-ion cathodes, a learned surrogate over synthesis descriptors enabled inverse ``retrosynthesis'' of processing conditions achieving high discharge capacity, validated experimentally~\cite{Liow2022_NanoEnergy}. Transfer learning further bolsters small-data surrogates; e.g., TLOpt leverages pretraining and then couples the surrogate to genetic and Bayesian optimizers to match target spectra in optical materials~\cite{Dong2021_CMS}. 
Finally, \emph{generative models} (VAEs, GANs, autoregressive flows) map low-dimensional latent variables to design manifolds, providing smooth spaces for search and conditional sampling. VAEs have been used to traverse continuous microstructure spaces to optimize mechanical response~\cite{Kim2021_MatDes}, and generative pipelines have supported inverse design of high-entropy refractory alloys~\cite{Debnath2021_JMI}. GAN-based inverse mappers have also produced morphing composite beams that realize prescribed shapes under actuation~\cite{Brzin2024_EAAI}. Model-based methods excel in sample efficiency and amortized inference once trained, but must explicitly manage non-uniqueness (e.g., via forward validators or conditional priors), distribution shift, and uncertainty quantification.

\subsubsection{Optimization-based inverse design.}
Here the inverse task is posed as optimizing one or more objectives (e.g., error to a target spectrum) over design variables, subject to constraints. \emph{Bayesian Optimization} (BO) offers strong sample efficiency by balancing exploration and exploitation with acquisition functions; in combination with graph deep-learning energy models, BO has led to experimentally realized superhard compounds and efficient exploration of hypothetical crystals~\cite{Zuo2021_MaterialsToday}. BO/active-learning loops have also delivered high-performance Mg--Mn alloys with few experiments~\cite{Mi2024_JMagAlloys}. Gaussian/Bayesian hybrids can tackle metamaterial inverse design with uncertainty-aware updates~\cite{Zheng2020_JAP}, and deep-learning Bayesian frameworks support attribute-driven molecular design under priors and constraints~\cite{Tagade2019_NPJCM}. 
\emph{Genetic algorithms} (GA) remain a strong baseline for discrete or constrained spaces and can be hybridized with surrogates; e.g., GA+NN co-optimization achieved efficient photonic device design with reduced data needs~\cite{Ren2021_PhotonicsResearch}. When full-wave or continuum physics is differentiable, \emph{topology optimization} (adjoint gradients) provides a powerful PDE-constrained inverse design tool widely adopted in photonics~\cite{Christiansen2021_JOSAB,Jensen2011_LPR}. Optimization-based schemes provide principled multi-objective trade-offs, constraint handling, and sample-efficient search, but depend on accurate surrogates/acquisitions, careful problem parameterization, and (for gradient methods) differentiable solvers and single-valued adjoints.

In practice, hybrid pipelines perform best: exploration agents query model-based surrogates (\emph{RL $\leftrightarrow$ surrogate}); generative latents are optimized with BO/GA (\emph{gen. model $\rightarrow$ optimizer}); and forward–inverse couplings screen P2C proposals through trusted C2P evaluators before simulation or experiment. Which path to prefer depends on data regime, simulator cost, constraints, and whether gradients are available. Table~\ref{tab:inv_compare} summarizes the trade-space.

\begin{table}[H]
\centering
\caption{Compact comparison of inverse-design strategies.}
\label{tab:inv_compare}
\setlength{\tabcolsep}{4pt}           % tighten horizontal padding
\renewcommand{\arraystretch}{1.12}    % subtle vertical compaction
\footnotesize                         % or \scriptsize if still wide
\begin{tabularx}{\linewidth}{
  >{\raggedright\arraybackslash}p{2.6cm}
  >{\raggedright\arraybackslash}X
  >{\raggedright\arraybackslash}X
  >{\raggedright\arraybackslash}X}
\toprule
\textbf{Strategy} & \textbf{Typical algorithms} & \textbf{Strengths} & \textbf{Common caveats} \\
\midrule
\emph{Exploration-based}
& RL (A2C/PPO + simulators); MCTS (discrete/continuous); PSO and other swarm heuristics
& Label-frugal; discovers diverse solutions; robust to one-to-many mappings
& Many simulator calls; reward-shaping sensitivity; constraint enforcement can be nontrivial \\
\addlinespace[2pt]
\emph{Model-based}
& Forward--inverse (tandem) nets; supervised surrogates (RF/NN); transfer learning; VAEs/GANs
& Sample-efficient; amortized inference; possible uncertainty handling via calibrated models
& Non-uniqueness requires forward validation; distribution shift; data curation and retraining loops \\
\addlinespace[2pt]
\emph{Optimization-based}
& Bayesian optimization / active learning; genetic algorithms; adjoint (topology) optimization
& Strong sample-efficiency (BO); handles constraints and multi-objective trade-offs; scalable with gradients
& Surrogate bias; local minima; need differentiable solvers/valid adjoints; parameterization sensitivity \\
\bottomrule
\end{tabularx}
\end{table}




\section{Modeling Background}\label{sec:modelling_background}

Inverse exploration relies on models as its inner loop: given candidate decisions $\vect{x}$, we must anticipate outcomes $\vect{y}$, and—crucially for design—quantify how confident we are in those predictions. Two complementary perspectives are standard. A \emph{deterministic} view learns a single mapping $f:\mathcal{X}\!\to\!\mathcal{Y}$ that returns point estimates (useful for speed and direct optimization). A \emph{probabilistic} view learns conditional distributions $p(\vect{y}\mid\vect{x})$ (and, when needed, posteriors over parameters or designs), making uncertainty an explicit object of reasoning. Both sit naturally within risk-minimization and decision-theoretic principles and will be used throughout our inverse workflows~\cite{Vapnik1998,Bishop2006,Murphy2012,HTF2009,Berger1985}.


\subsection{Learning as Risk Minimization}

Let $\mathcal{D}=\{(\vect{y}_i,\vect{x}_i)\}_{i=1}^{N}$ denote observed outcome–decision pairs. In inverse exploration we seek a function $g:\mathcal{Y}\!\to\!\mathcal{X}$ that proposes decisions $g(\vect{y})$ aligned with a target outcome pattern. This can be framed by the (unknown) expected risk
\begin{equation}
  \mathcal{R}(g)
  \;=\;
  \mathbb{E}_{(\vect{y},\vect{x})}\!\left[\,
    \ell\!\big(g(\vect{y}),\,\vect{x}\big)
  \right],
  \label{eq:true_risk}
\end{equation}
where $\ell(\cdot,\cdot)$ is task-appropriate (e.g., squared error for regression-style inverse mappings, or constraint-aware penalties when feasibility must be respected)~\cite{Hastie2009ESL,Bishop2006PRML}.

As $\mathcal{R}(g)$ is not observable, we minimise a regularised empirical surrogate:
\begin{equation}
  \widehat{g}
  \;\in\;
  \arg\min_{g\in\mathcal{G}}
  \underbrace{\frac{1}{N}\sum_{i=1}^{N}\ell\!\left(g(\vect{y}_i),\,\vect{x}_i\right)}_{\text{empirical risk}}
  \;+\;
  \lambda\,\underbrace{\Omega(g)}_{\text{regularisation}}\,,
  \label{eq:erm}
\end{equation}
with hypothesis class $\mathcal{G}$ (e.g., linear models, kernels, neural networks), regulariser $\Omega$ (e.g., smoothness, sparsity, complexity control), and trade-off $\lambda\!\ge\!0$~\cite{Hastie2009ESL,Bishop2006PRML}. Regularisation improves generalisation and stabilises inherently ill-posed inverses.

\emph{Forward-consistency when a forward model is available.} If prior knowledge of a forward map $\vect{f}$ exists, we can couple inverse and forward information so that $g(\vect{y})$ proposes decisions whose forward evaluation is consistent with $\vect{y}$:
\begin{equation}
  \widehat{g}
  \;\in\;
  \arg\min_{g\in\mathcal{G}}
  \frac{1}{N}\sum_{i=1}^{N}
  \Big[
    \alpha\,\ell\!\left(g(\vect{y}_i),\,\vect{x}_i\right)
    \;+\;
    (1-\alpha)\,\mathcal{L}\!\left(\vect{f}\!\big(g(\vect{y}_i)\big),\,\vect{y}_i\right)
  \Big]
  \;+\; \lambda\,\Omega(g),
  \quad \alpha\in[0,1],
  \label{eq:hybrid}
\end{equation}
where $\mathcal{L}$ penalises discrepancy in outcome space. This blends purely data-driven inverse learning with physics-/simulation-aware consistency~\cite{Bishop2006PRML,Murphy2022PML1}.

\textbf{Representation, inductive bias, and constraints.}
Our goal is to learn $g:\mathcal{Y}\!\to\!\mathcal{X}$ that proposes feasible, credible decisions for target outcomes. Three design axes govern performance: (i) \emph{representation} (the hypothesis class $\mathcal{G}$), (ii) \emph{inductive bias} (regularisation and architectural choices that stabilise ill-posed inverses), and (iii) \emph{feasibility handling} (how domain constraints are enforced). These choices control not only in-sample fit but also robustness, out-of-support behaviour, and the diversity/credibility of proposed decisions~\cite{Bishop2006PRML,Hastie2009ESL,Murphy2022PML1}. Deterministic and probabilistic formulations both fit naturally within a risk-minimisation or likelihood-based view, with evaluation by predictive performance and, when distributions are produced, proper scoring rules and decision-theoretic criteria.

\subsection{Deterministic Modelling}\label{sec:deterministic_modelling}
In the deterministic view, the inverse model returns a \emph{single} representative decision for each target outcome. Learning $g:\mathcal{Y}\!\to\!\mathcal{X}$ is posed as regularised ERM:
\begin{equation}
  \widehat{g}
  \;\in\;
  \arg\min_{g\in\mathcal{G}}
  \frac{1}{N}\sum_{i=1}^{N}\ell\!\big(g(\vect{y}_i),\,\vect{x}_i\big)
  \;+\; \lambda\,\Omega(g),
  \label{eq:erm_det}
\end{equation}
with task loss (e.g., $\ell(\hat{\vect{x}},\vect{x})=\|\hat{\vect{x}}-\vect{x}\|_2^2$) and inductive bias via $\Omega(g)$ (e.g., weight decay, Lipschitz/Jacobian penalties, early stopping) to curb overfitting and to stabilise estimates under non-uniqueness and noisy data. Hyperparameters, including $\lambda$, are tuned by validation or cross-validation to balance bias–variance and ensure generalisation to unseen queries $\vect{y}^\star$.

\begin{figure}[tbp]
  \centering
  \begin{adjustbox}{max width=\linewidth}
    \begin{tikzpicture}
  % ===== Local colors & styles (only for this picture) =====
  \definecolor{detBlueA}{RGB}{33,111,219}
  \definecolor{softGrayA}{RGB}{246,248,250}

  % --- All styles are defined BEFORE any draws ---
  \tikzset{
    >={Latex[length=2.4mm]},       
    font=\sffamily,

    % Panel style (auto-sized via fit below)
    panelA/.style={
      draw, rounded corners=8pt, line width=0.8pt, fill=detBlueA!6
    },

    % Wrapped node styles
    boxA/.style={
      draw, rounded corners=6pt, line width=0.8pt, fill=white,
      inner sep=7pt, minimum height=10mm,
      align=center, text width=35mm
    },
    smallA/.style={boxA, text width=22mm},
    wideA/.style={boxA, text width=35mm},

    noteA/.style={
      align=left, font=\scriptsize, inner sep=1pt, text opacity=0.85,
      text width=12.2cm
    },
    circA/.style={
      circle, draw, fill=white, line width=0.8pt, minimum size=8mm,
      align=center, text width=8mm
    },

    % Flow arrows (define BEFORE first use)
    flowA/.style={-Latex, line width=0.9pt},

    softlabelA/.style={
      fill=softGrayA, text=black, rounded corners=3pt,
      inner sep=1.2pt, font=\scriptsize, align=center, text width=22mm
    }
  }

  %------------------------------------------------------------------
  % CONTENT: bounded so the panel can fit everything
  %------------------------------------------------------------------
  \begin{scope}[local bounding box=contentA]
    % ===== Nodes =====
    \node[smallA] (ystar) at (-5.6,0.9) {$\mathbf{y}^{\star}$\\{\scriptsize target}};
    \node[wideA, right=14mm of ystar] (g) {$\hat{g}\!:\;\mathcal{Y}\!\to\!\mathcal{X}$\\{\scriptsize regularised ERM}};
    \node[smallA, right=16mm of g] (xhat) {$\hat{\mathbf{x}}$\\{\scriptsize candidate}};
    \node[wideA, right=16mm of xhat] (fwd) {Forward model $f$ / $\,\hat f$\\{\scriptsize verification}};
    \node[smallA, right=16mm of fwd] (yhat) {$\hat{\mathbf{y}}$\\{\scriptsize predicted}};

    % Feasibility block (wrapped)
    \node[boxA, text width=39mm, below=9mm of xhat] (feas)
      {Feasibility\\{\scriptsize project/penalise}\\[1pt]
       $g_j(\hat{\mathbf{x}})\!\le\!0,\;h_k(\hat{\mathbf{x}})\!=\!0$};

    % ===== Flows =====
    \draw[flowA] (ystar) -- node[above,sloped,softlabelA]{query} (g);
    \draw[flowA] (g) -- node[above,softlabelA]{single output} (xhat);
    \draw[flowA] (xhat) -- (fwd);
    \draw[flowA] (fwd) -- node[above,softlabelA]{$e_f=\|\hat{\mathbf{y}}-\mathbf{y}^{\star}\|$} (yhat);

    % Feasibility loop
    \draw[flowA] (xhat.south) -- ++(0,-4mm) -| (feas.north);
    \draw[flowA] (feas.east) -| ++(15mm,0) |- (xhat.east);
  \end{scope}

  %------------------------------------------------------------------
  % AUTO-SIZED PANEL behind everything
  %------------------------------------------------------------------
  \begin{scope}[on background layer]
    \node[panelA, fit=(contentA), inner sep=12pt] (P) {};
  \end{scope}

  % Tight bounding box = exactly the panel rectangle
  \path[use as bounding box] (P.south west) rectangle (P.north east);

  % Title anchored to the fitted panel
  \node[anchor=north west, font=\bfseries\large]
       at ([xshift=6pt,yshift=-6pt]P.north west) {Deterministic Modelling};
\end{tikzpicture}

  \end{adjustbox}
  \caption{Deterministic modelling pipeline with regularised ERM and feasibility handling.}
  \label{fig:deterministic-model}
\end{figure}

When decisions must satisfy feasibility constraints $g_j(\vect{x})\!\le\!0$ and $h_k(\vect{x})\!=\!0$
(cf.\ Eq.~\eqref{eq:mop}), training can promote feasibility either by projection onto a feasible set
$\Pi_{\calX}$ or via soft penalties:
\begin{equation}
  \widehat{g}
  \;\in\;
  \arg\min_{g}
  \frac{1}{N}\sum_{i=1}^{N}
  \Big[
     \ell\!\big(\Pi_{\calX}(g(\vect{y}_i)),\,\vect{x}_i\big)
     \;+\;
     \beta\!\sum_{j}\![\,g_j(g(\vect{y}_i))\,]_+^{\!2}
     \;+\;
     \beta'\!\sum_{k}\! h_k(g(\vect{y}_i))^{2}
  \Big]
  \;+\; \lambda\,\Omega(g),
  \label{eq:constraints_det}
\end{equation}
with $[t]_+=\max\{t,0\}$. Such Tikhonov-style penalties discourage implausible or infeasible proposals while keeping the output single-valued.

Deterministic outputs are well-suited for fast, repeatable proposals—e.g., as starting points for expert refinement or forward verification—yet they \emph{collapse} potentially multi-valued inverse relations to one point and do not natively quantify uncertainty or capture diversity. This motivates the complementary probabilistic view.

\subsection{Probabilistic Modelling}

The probabilistic view represents \emph{distributions} conditional on inputs, thereby quantifying uncertainty and naturally handling multi-valued relationships~\cite{Murphy2022PML1,Gelman2013BDA3}. A conditional model $p_\theta(\vect{y}\mid\vect{x})$ can be fit by maximising regularised conditional log-likelihood:
\begin{equation}
  \widehat{\theta}
  \;\in\;
  \arg\max_{\theta}
  \bigg[
  \sum_{i=1}^{N}\log p_{\theta}\!\big(\vect{y}_i \mid \vect{x}_i\big)
  \;-\; \gamma\,\mathcal{R}(\theta)
  \bigg],
  \label{eq:mle}
\end{equation}
where $\mathcal{R}(\theta)$ controls complexity. Given a new input $\vect{x}^{\star}$, predictive summaries include
\begin{align}
  \text{(point)}\quad
  \widehat{\vect{y}}_{\text{MAP}}(\vect{x}^{\star}) 
  &= \arg\max_{\vect{y}}\, p_{\widehat{\theta}}(\vect{y}\mid \vect{x}^{\star}),
  \label{eq:map_point}\\[2pt]
  \text{(predictive)}\quad
  p(\vect{y}\mid \vect{x}^{\star},\mathcal{D})
  &= \int p_{\theta}(\vect{y}\mid \vect{x}^{\star})\,
     p(\theta\mid\mathcal{D})\,\mathrm{d}\theta
     \;\approx\; p_{\widehat{\theta}}(\vect{y}\mid \vect{x}^{\star}),
  \label{eq:predictive}
\end{align}
where the Bayesian predictive integrates parameter uncertainty whereas the plug-in predictive uses $\widehat{\theta}$~\cite{Gelman2013BDA3,Murphy2022PML1}. In inverse design one may also model $p(\vect{x}\mid\vect{y})$ directly (e.g., conditional generative models), or combine a learned $p(\vect{y}\mid\vect{x})$ with decision rules to propose $\vect{x}$ that are likely to realise $\vect{y}^{\star}$.

\begin{figure}[tbp]
  \centering
  \begin{adjustbox}{max width=\linewidth}
    \def\laneSep{20mm} % vertical gap between training and inference lanes

\begin{tikzpicture}[node distance=12mm and 14mm]
  \definecolor{panelGreen}{RGB}{30,145,85}
  \definecolor{softGray}{RGB}{246,248,250}

  \tikzset{
    >={Latex[length=2.6mm]},
    font=\sffamily,
    panel/.style={draw, rounded corners=8pt, line width=0.8pt, fill=panelGreen!6},
    box/.style={draw, rounded corners=6pt, line width=0.8pt, fill=white,
                align=center, inner sep=7pt, minimum height=10mm, text width=34mm},
    small/.style={box, text width=24mm},
    flow/.style={-Latex, line width=0.9pt},
    circ/.style={circle, draw, fill=white, line width=0.8pt, align=center, minimum size=8mm},
    chip/.style={fill=softGray, rounded corners=3pt, inner sep=1.2pt,
                 font=\scriptsize, align=center}
  }

  % ===================== CONTENT =====================
  \begin{scope}[local bounding box=content]

    % -------- TRAINING --------
    \node[small]                         (xtr)   at (-6,1.4) {$\mathbf{X}$\\{\scriptsize features}};
    \node[box,   right=of xtr]           (pyx)   {$p_\theta(\mathbf{y}\mid \mathbf{X})$\\{\scriptsize conditional model}};
    \node[box,   right=of pyx]           (nll)   {Objective\\{\scriptsize minimize NLL}\\[-2pt]{\scriptsize $-\!\sum\log p_\theta(\mathbf{y}\mid\mathbf{X})$}};
    \node[small, below=12mm of nll]      (ytr)   {$\mathbf{y}$\\{\scriptsize targets}};

    \draw[flow] (xtr) -- node[above, chip]{condition on $\mathbf{X}$} (pyx);
    \draw[flow] (pyx) -- (nll);
    \draw[flow] (ytr) -- node[right, chip]{supervision} (nll.south);
    \draw[flow] (nll.south) |- ++(-32mm,-8mm) -| node[pos=0.25, below, chip]{update $\theta$} (pyx.south);

    \node[anchor=west, font=\bfseries] at ([yshift=2mm]xtr.north west) {Training (MLE/MAP)};

    % Fit node to capture true bounds of the training lane
    \node[inner sep=0, fit=(xtr)(pyx)(nll)(ytr), name=trainLane] {};

    % Reference y-position for the inference lane: \laneSep below training lane
    \coordinate (laneY) at ([yshift=-\laneSep]trainLane.south);

    % -------- INFERENCE --------
    % Place first inference node aligned in x with xtr, at y = laneY
    \node[small] at ($(xtr|-laneY)$)    (xnew)  {$\mathbf{X}^{\!*}$\\{\scriptsize new features}};
    \node[box,   right=of xnew]         (pyx2)  {$p_{\hat\theta}(\mathbf{y}\mid \mathbf{X}^{\!*})$\\{\scriptsize trained model}};
    \node[circ,  right=14mm of pyx2, yshift=8mm, minimum size=7.5mm] (s1) {$\mathbf{y}_1$};
    \node[circ,  right=14mm of pyx2,               minimum size=7.5mm] (s2) {$\mathbf{y}_2$};
    \node[circ,  right=14mm of pyx2, yshift=-8mm,  minimum size=7.5mm] (s3) {$\mathbf{y}_3$};
    \node[box,   right=16mm of s2, text width=40mm] (dec)
          {Decision rule\\{\scriptsize MAP / posterior mean / quantile}};
    \node[small, right=of dec]          (yhat)  {$\hat{\mathbf{y}}$\\{\scriptsize prediction}};

    \draw[flow] (xnew) -- (pyx2);
    \foreach \s in {s1,s2,s3}{
      \draw[flow] (pyx2.east) -- ++(6mm,0) |- (\s.west);
      \draw[flow] (\s.east) -- (dec.west);
    }
    \draw[flow] (dec) -- (yhat);

    \node[anchor=west, font=\bfseries] at ([yshift=9mm]xnew.north west) {Inference};
  \end{scope}

  % ===================== PANEL, SEPARATOR & TITLE =====================
  \begin{scope}[on background layer]
    \node[panel, fit=(content), inner sep=12pt] (P) {};
  \end{scope}

  \path[use as bounding box] (P.south west) rectangle (P.north east);
  
  \node[anchor=north west, font=\bfseries\large]
        at ([xshift=6pt,yshift=-6pt]P.north west) {Probabilistic Supervised Modelling};
\end{tikzpicture}

  \end{adjustbox}
  \caption{Probabilistic modelling via conditional density $p_\theta(\vect{y}\mid\vect{x})$, sampling, and decision rule.}
  \label{fig:probabilistic-model}
\end{figure}

\paragraph{Aleatoric vs.\ epistemic uncertainty.}
Probabilistic models distinguish data (aleatoric) uncertainty from model (epistemic) uncertainty: the former reflects intrinsic variability/noise in $\vect{y}\mid\vect{x}$, the latter arises from limited data or misspecification. This distinction matters when ranking candidates and communicating confidence~\cite{Murphy2022PML1,Bishop2006PRML}.

\subsection{Decision-Theoretic Perspective}

For inverse exploration the object we report is a \emph{decision} $\tilde{\vect{x}}\in\calX$ for a given target outcome $\vect{y}^\star$. The decision quality is determined by the (uncertain) outcome that $\tilde{\vect{x}}$ will induce via the forward map $\vect{f}$. Under a task utility $u(\tilde{\vect{x}},\vect{y};\,\vect{y}^\star)$—often chosen as a negative discrepancy between $\vect{f}(\tilde{\vect{x}})$ and $\vect{y}^{\star}$ subject to feasibility—the Bayes rule selects
\begin{equation}
  \tilde{\vect{x}}^{\star}
  \;\in\;
  \arg\max_{\tilde{\vect{x}}\in\calX}
  \mathbb{E}_{\vect{y}\sim p(\vect{y}\mid \tilde{\vect{x}},\mathcal{D})}
  \big[\,u(\tilde{\vect{x}},\vect{y};\,\vect{y}^{\star})\,\big],
  \label{eq:bayes_decision_inverse}
\end{equation}
which minimises expected loss in the equivalent formulation. Canonical choices illustrate the link to summaries of the predictive distribution: with quadratic loss on $d(\vect{f}(\tilde{\vect{x}}),\vect{y}^\star)$, the \emph{mean} predictive outcome is optimal; with absolute deviation, the \emph{median} is optimal; with a tolerance region $\{\vect{y}\!:\, d(\vect{y},\vect{y}^\star)\le\epsilon\}$, decisions can be ranked by success probability $\Pr\!\big[d(\vect{y},\vect{y}^\star)\le\epsilon\,|\,\tilde{\vect{x}}\big]$. This connects probabilistic predictors to point recommendations and makes uncertainty operational in the inverse task~\cite{Gelman2013BDA3,Murphy2022PML1}.

\medskip
Importantly, Eq.~\eqref{eq:bayes_decision_inverse} is \emph{model-agnostic}: one may plug in a forward predictive $p(\vect{y}\mid\vect{x})$ (paired with a search over $\vect{x}$) or a direct inverse density $p(\vect{x}\mid\vect{y})$ (paired with sampling/selection at $\vect{y}^\star$). Either way, the decision layer ensures that reported designs optimise a task-aligned utility rather than a surrogate training loss.

\subsection{Deterministic vs.\ probabilistic objectives}

\textbf{Deterministic (point) objectives.}
Deterministic models learn a single-valued inverse $g:\calY\!\to\!\calX$ by regularised ERM,
\begin{equation}
  \widehat{g}
  \;\in\;
  \arg\min_{g\in\calG}
  \frac{1}{N}\sum_{i=1}^{N}\ell\!\big(g(\vect{y}_i),\,\vect{x}_i\big)
  \;+\; \lambda\,\Omega(g),
  \label{eq:erm_det_again}
\end{equation}
with $\Omega$ injecting inductive bias (smoothness, sparsity, Lipschitz/Jacobian control, early stopping). Deterministic training is attractive for speed, stability, and easy feasibility handling (e.g., projections or soft penalties), but collapses multi-valued inverse relations to a point and does not natively quantify uncertainty~\cite{Bishop2006PRML,Hastie2009ESL}.

\textbf{Probabilistic (distributional) objectives.}
There are two common routes:
(i) \emph{inverse density} learning, which directly fits $p_\theta(\vect{x}\mid\vect{y})$ by maximising conditional log-likelihood (a strictly proper scoring rule),
\begin{equation}
  \widehat{\theta}
  \;\in\;
  \arg\max_{\theta}\;
  \sum_{i=1}^{N}\log p_{\theta}\!\big(\vect{x}_i\mid \vect{y}_i\big)
  \;-\;\gamma\,\mathcal{R}(\theta),
  \label{eq:cond_nll}
\end{equation}
or (ii) \emph{forward density} learning, which fits $p_\theta(\vect{y}\mid\vect{x})$ via \eqref{eq:mle} and then applies the decision rule \eqref{eq:bayes_decision_inverse} to choose $\tilde{\vect{x}}$ that is likely to realise $\vect{y}^\star$. Route (ii) is convenient when forward simulators or validated surrogates already exist; route (i) enables direct sampling of diverse candidates at $\vect{y}^\star$. In both cases, likelihood-based training yields calibrated uncertainty when paired with proper scoring rules and well-specified families~\cite{Murphy2022PML1,Gelman2013BDA3,Gneiting2007JASA}.

\textbf{Bridging the two.}
Hybrid objectives combine inverse consistency with forward consistency, e.g., \eqref{eq:hybrid}, or train a probabilistic model and output a deterministic recommendation via the Bayes rule (plug-in decision on the predictive). This separation—\emph{train for distributional fidelity, decide for task utility}—keeps learning and decision-making aligned yet modular~\cite{Bishop2006PRML,Murphy2022PML1}.

\subsection{Inverse-centric evaluation metrics}

Evaluation should reflect the inverse goal—“propose decisions that realise $\vect{y}^{\star}$ while respecting constraints”—and should match the model type (point vs.\ distribution). We therefore report complementary, task-aligned criteria:

\paragraph{Target attainment in outcome space.}
For a proposed decision $\widehat{\vect{x}}$, measure discrepancy in $\calY$:
\begin{equation}
  \mathcal{E}_{\mathcal{Y}}(\widehat{\vect{x}},\vect{y}^{\star})
  \;=\;
  d\!\left(\,\vect{f}(\widehat{\vect{x}})\,,\,\vect{y}^{\star}\right),
  \label{eq:target_error_clean}
\end{equation}
with a task distance $d$ (e.g., $\ell_2$, Mahalanobis, or application-specific metrics). Report also a tolerance-based success rate
\begin{equation}
  \mathrm{Succ}_\epsilon
  \;=\;
  \frac{1}{M}\sum_{m=1}^{M}
  \mathbb{I}\!\left[\,d\!\left(\vect{f}\!\big(\widehat{\vect{x}}^{(m)}\big),\,\vect{y}^{\star (m)}\right)\le \epsilon\,\right],
  \label{eq:succ_eps}
\end{equation}
and, when an oracle or best-known candidate set $\calC$ is available, an \emph{outcome regret}
\begin{equation}
  \mathrm{Regret}
  \;=\;
  \frac{1}{M}\sum_{m=1}^{M}\Big(
  d\!\left(\vect{f}(\widehat{\vect{x}}^{(m)}),\,\vect{y}^{\star(m)}\right)
  - \min_{\vect{x}\in\calC} d\!\left(\vect{f}(\vect{x}),\,\vect{y}^{\star(m)}\right)
  \Big).
  \label{eq:regret}
\end{equation}
When full simulators are costly, use a validated surrogate $d_{\text{surf}}$ with explicit caveats.

\paragraph{Feasibility and constraint violation.}
Report both the feasible rate and the average violation magnitude:
\begin{equation}
  \mathrm{Feas}=\frac{1}{M}\sum_{m}\mathbb{I}\!\big[g_j(\widehat{\vect{x}}^{(m)})\!\le\!0,\ \forall j\ \wedge\ h_k(\widehat{\vect{x}}^{(m)})\!=\!0,\ \forall k\big],
  \qquad
  \mathrm{Viol}=\frac{1}{M}\sum_{m}\Big(\sum_j[\,g_j(\widehat{\vect{x}}^{(m)})\,]_+ + \sum_k |h_k(\widehat{\vect{x}}^{(m)})|\Big).
  \label{eq:feas_viol}
\end{equation}

\paragraph{Decision-space diversity (when multiple proposals are returned).}
For $\{\widehat{\vect{x}}^{(m,r)}\}_{r=1}^{R}$ per target, quantify spread, e.g., by average pairwise distance
\begin{equation}
  \mathrm{Div}^{(m)}
  \;=\;
  \frac{2}{R(R-1)}\sum_{r<s}\big\|\widehat{\vect{x}}^{(m,r)}-\widehat{\vect{x}}^{(m,s)}\big\|_2,
  \label{eq:diversity_clean}
\end{equation}
or a problem-specific dispersion index. Diversity should be reported \emph{alongside} target attainment to avoid trading quality for spread.

\paragraph{Uncertainty quality (probabilistic models).}
Use proper scoring rules on held-out $(\vect{y}_i,\vect{x}_i)$ pairs, such as negative log-likelihood for $p(\vect{x}\mid\vect{y})$ or $p(\vect{y}\mid\vect{x})$, and CRPS for continuous targets; these jointly reward calibration and sharpness~\cite{Gneiting2007JASA}. When ranking decisions via success probabilities, reliability diagrams or PIT diagnostics complement scalar scores.

\paragraph{Efficiency.}
Report simulator/forward calls and wall-clock per query $\vect{y}^{\star}$, enabling fair comparisons across models, datasets, and hardware.

\medskip
Together, these criteria separate \emph{training} (deterministic vs.\ probabilistic objectives) from \emph{reporting} (decision-theoretic selection under uncertainty), and evaluate models on what matters for inverse design: achieving $\vect{y}^\star$, respecting constraints, being honest about uncertainty, and doing so efficiently.
