%-------------------------------------------------------------------------------
% Methodology
%-------------------------------------------------------------------------------

\chapter{Methodology}\label{chap:method}

This chapter presents the methodology developed to answer the research
questions posed in Chapter~\ref{chap:problem}.  The approach is modular
and consists of an offline training phase and an online exploration
phase.  During the offline phase we generate a Pareto approximation
through a standard MOO algorithm, train surrogate forward and inverse
models and validate their performance.  In the online phase a user
interacts with the inverse model to specify desired objectives and
retrieve candidate decisions, which are then ranked using forward
predictions and plausibility scores.

\section{Offline Phase: Data Generation and Model Training}

\subsection{Generation of Pareto Samples}

The first step is to obtain a representative set of Pareto‑optimal or
near‑optimal solutions.  Any suitable multi‑objective optimisation
algorithm can be employed; in our experiments we use the Non‑dominated
Sorting Genetic Algorithm II (NSGA‑II) and the Multi‑Objective
Evolutionary Algorithm based on Decomposition (MOEA/D) as baselines.  The
resulting archive \(\mathcal{D}=\{(\vect{x}_i,\vect{y}_i)\}_{i=1}^N\)
contains decision vectors \(\vect{x}_i\in\calX\) and their objective
values \(\vect{y}_i=\vect{f}(\vect{x}_i)\).  To ensure diversity and
coverage of the Pareto front the optimisation is run with standard
population sizes and termination criteria, and the union of final
non‑dominated solutions over multiple runs is stored.

\subsection{Data Preprocessing}

Prior to model training we normalise the objective vectors to lie in
\([0,1]^m\) using min–max scaling.  This step ensures that objectives
with different units or ranges contribute equally to the inverse model.
Similarly, decision variables are scaled to \([0,1]^n\) to avoid biasing
the regressors.  Outliers are removed using interquartile range
thresholding.  The processed dataset is then randomly split into
training, validation and test sets.

\subsection{Inverse Model Training}

We consider several modelling strategies for the inverse mapping
\(\hat{g}\colon\calY\to\calX\):

\paragraph{Gaussian Process Regression (GPR).}  GPR models the inverse
map as a collection of independent Gaussian processes, one for each
decision variable.  Given training pairs \((\vect{y}_i,x_{i,j})\), the
process defines a mean function and covariance kernel over the objective
space.  Hyperparameters of the kernel (e.g. length scales and noise
variance) are optimised by maximising the marginal likelihood.  The
advantage of GPR is its ability to quantify prediction uncertainty and to
capture both smooth trends and local irregularities~\cite{Farias2024}.

\paragraph{Radial Basis Function Networks (RBF).}  An RBF network
comprises a hidden layer of basis functions \(\phi_k(\vect{y})=\exp(-\|\vect{y}-\vect{c}_k\|^2/(2\sigma_k^2))\) and a linear output layer.
Centres \(\vect{c}_k\) and bandwidths \(\sigma_k\) are chosen by k‑means
clustering and heuristics; weights are solved by linear least squares.
RBFs are universal approximators and perform well with moderate amounts
of data.

\paragraph{Random Forest Regression (RF).}  RF ensembles multiple decision
trees grown on bootstrap samples of the training data.  Each tree
predicts a decision variable as a function of the objective vector.  The
final prediction is the average over trees.  Random forests are robust to
noise and can handle high‑dimensional inputs.  They have been shown to
improve diversity and convergence in inverse evolutionary algorithms~\cite{Gholamnezhad2022}.

\paragraph{Feed‑Forward Neural Networks (FFNN).}  A multi‑layer perceptron
with fully connected layers and ReLU activations can approximate complex
nonlinear mappings.  We select the network architecture using
hyperparameter search and train using Adam optimisation with early
stopping.  Data augmentation and regularisation (dropout and weight
decay) improve generalisation.

\paragraph{Transfer Learning and Multi‑Source Data.}  When the available
dataset is small, transfer learning can leverage data from related tasks.
Tan et~al. propose learning across common objective spaces to mitigate
data scarcity and demonstrate that combining multiple Pareto fronts can
  improve inverse model generalisation~\cite{Tan2023}.  We adopt
this idea by pooling Pareto samples from benchmark problems with similar
objective definitions during training.

\subsection{Model Selection and Validation}

For each modelling strategy we perform hyperparameter tuning via
k‑fold cross‑validation on the training set.  The loss function combines
the mean squared error between predicted and true decision variables and a
penalty on predicted values outside the feasible range.  Models are
compared using performance metrics such as root mean squared error
(RMSE), coefficient of determination \(R^2\) and the generational
distance between the reconstructed Pareto front and the true front.  We
select the model with the best validation performance and retain it for
the online phase.

\subsection{Forward Surrogate Model}

In addition to the inverse model we train a surrogate forward model
\(\hat{f}\colon\calX\to\calY\) on the same data.  This model is used
during the online phase to rapidly evaluate candidate decisions without
incurring the cost of running the underlying simulation or experiment.  A
Gaussian process is employed due to its good performance on small data;
alternatively, a neural network or polynomial regression can be used.

\section{Online Phase: Interactive Exploration}

During the online phase the trained inverse and forward models are used
to support interactive design exploration.  The process comprises the
following steps:

\paragraph{Target specification.}  The user specifies a desired objective
vector \(\vect{y}^\star\) either numerically or by selecting a point in a
visualisation of the objective space.  Input widgets allow users to
adjust individual objectives and observe how the target moves relative
to the sampled Pareto front.

\paragraph{Plausibility check.}  The system computes the distance
\(d(\vect{y}^\star, \mathcal{D})\) to the nearest sample in the objective
space and derives a plausibility score.  If the score is below a
user‑defined threshold, a warning is displayed indicating that the
desired trade‑off may not be achievable.

\paragraph{Inverse prediction.}  The inverse model \(\hat{g}\) is queried
with \(\vect{y}^\star\) to produce a set of candidate decisions
\(\{\hat{\vect{x}}^{(k)}\}_{k=1}^K\).  For probabilistic models such as
GPR we draw multiple samples from the posterior distribution; for
deterministic models such as RBFs and FFNNs we sample by adding small
Gaussian noise to the input.

\paragraph{Forward evaluation.}  Each candidate is evaluated using the
forward surrogate \(\hat{f}\) to compute the predicted objective vector
\(\hat{\vect{y}}^{(k)}=\hat{f}(\hat{\vect{x}}^{(k)})\).  The forward
error \(e_f^{(k)}=\|\hat{\vect{y}}^{(k)}-\vect{y}^\star\|\) and distance
to the sampled Pareto front are computed.

\paragraph{Ranking and selection.}  Candidates are ranked according to a
multi‑criteria score combining forward error, diversity and plausibility.
The top‑\(M\) candidates are presented to the user along with their
predicted objective values and a visual indication of their proximity to
the Pareto front.

\paragraph{User feedback.}  The user can select a candidate and, if
desired, refine the target or adjust weights in the ranking.  Selected
candidates can be saved for further simulation or fabrication.  User
interactions can be logged to adapt the inverse model over time.

\section{Algorithmic Summary}

Algorithm~\ref{alg:offline} summarises the offline training procedure and
Algorithm~\ref{alg:online} details the online exploration loop.  The
pseudocode abstracts away implementation details to highlight the main
steps.

\begin{algorithm}[t]
  \caption{Offline Training of Inverse and Forward Models}
  \label{alg:offline}
  \begin{algorithmic}[1]
    \Require multi‑objective function \(\vect{f}\), decision space \(\calX\)
    \Ensure trained inverse model \(\hat{g}\) and forward surrogate \(\hat{f}\)
    \State \textbf{Generate Pareto samples} \(\mathcal{D}=\{(\vect{x}_i,\vect{y}_i)\}\)
       by running a MOO algorithm (e.g. NSGA‑II)
    \State \textbf{Normalise} decision and objective vectors
    \State \textbf{Split} \(\mathcal{D}\) into training/validation/test sets
    \State \textbf{Select model type} (GP, RBF, RF, FFNN)
    \State \textbf{Tune hyperparameters} via cross‑validation
    \State \textbf{Train inverse model} \(\hat{g}\) on training set
    \State \textbf{Train forward model} \(\hat{f}\) on training set
    \State \textbf{Validate} on test set; select best model
    \State \Return \(\hat{g},\hat{f}\)
  \end{algorithmic}
\end{algorithm}

\begin{algorithm}[t]
  \caption{Online Inverse Decision Exploration}
  \label{alg:online}
  \begin{algorithmic}[1]
    \Require trained models \(\hat{g},\hat{f}\), dataset \(\mathcal{D}\)
    \Repeat
      \State \textbf{Get target} objectives \(\vect{y}^\star\) from user
      \State \textbf{Compute plausibility score} using distance to \(\mathcal{D}\)
      \If{score $<\tau$}
        \State display warning and optionally project \(\vect{y}^\star\) onto front
      \EndIf
      \State \textbf{Generate candidates} \(\{\hat{\vect{x}}^{(k)}\}\) via \(\hat{g}\)
      \For{each candidate \(k\)}
        \State evaluate \(\hat{\vect{y}}^{(k)}=\hat{f}(\hat{\vect{x}}^{(k)})\)
        \State compute forward error and distance metrics
      \EndFor
      \State \textbf{Rank} candidates by error, diversity and plausibility
      \State \textbf{Display} top candidates and solicit user feedback
    \Until{user terminates}
  \end{algorithmic}
\end{algorithm}

\section{Implementation Considerations}

Our implementation is written in Python and relies on the \texttt{scikit‑learn}
library for regression models and \texttt{pymoo} for multi‑objective
optimisation.  The training pipeline is modular so that additional
inverse models can be plugged in with minimal code changes.  Data
preprocessing, model training and cross‑validation are orchestrated via
\texttt{scikit‑learn}'s pipelines.  Hyperparameters are tuned using
\texttt{RandomizedSearchCV}.  The user interface is implemented in
\texttt{Dash} and provides sliders and plots for setting targets and
visualising candidates.  Parallelisation is exploited to evaluate
multiple candidates concurrently.